%------------------------
% Resume in Latex
%------------------------
% Author: 		Harshibar
% Based on:     https://github.com/jakegut/resume
% License:		MIT
% Edited by:	https://github.com/tobiplay (2024)
%------------------------
\documentclass[a4, 11pt]{article}  % Basic document class.

\usepackage{resume}  % Custom resume package with all styles.

%------------------------
% Resume content
%------------------------
\begin{document}
\sffamily  % Set sans-serif font.

%------------------------
% Header
%------------------------
\begin{tabular*}{\textwidth}{l@{\extracolsep{\fill}}r}
    {\textbf{\LARGE \color{\favColor}{John Doe}} \vspace{2pt}} &
    \\
    \small
    123 Main Street, Apt 4B, San Francisco, CA 94105\hspace{.5pt} |
    \hspace{.5pt}(123) 456-7890\hspace{.5pt} |
    \hspace{.5pt}{\textbf{\color{\favColor}\uline{\href{mailto:johndoe@example.com}{Mail}}}}\hspace{.5pt} |
    \hspace{.5pt}{\textbf{\color{\favColor}\uline{\href{https://www.linkedin.com/}{LinkedIn}}}}
\end{tabular*}

\justifying  % Justify the text.

%------------------------
% Education
%------------------------
\section{Education}

\resumeSubHeadingListStart
\resumeSubheading
{Princeton University}{Oct. 2021}
{Master of Science in Computer Science (GPA 3.7/4.0)}{}
\resumeSubheading
{Stanford University}{Sept. 2019}
{Bachelor of Science in Computer Science (GPA 3.6/4.0)}{}
\resumeSubHeadingListEnd

%------------------------
% Experience
%------------------------
\section{Experience}

\resumeSubHeadingListStart

\resumeSubheading
{Google LLC}{Jan. 2022 -- Present}
{Research Software Engineer}{Mountain View, CA}
\resumeItemListStart
\resumeItem{Reduced back-end's query response time by \qty{15}{\percent} via ML-driven "smart caching" (Python, PyTorch, Redis, Flask).}
\resumeItem{Developed a reinforcement learning algorithm (Python, PyTorch, MLflow) that increased ad click-through rates on category-B-websites by \qty{4}{\percent} by automatically optimizing ad placements.}
\resumeItem{Reduced deployment time of multiple Google-maintained open-source microservices by \qty{20}{\percent} by implementing scalable CI/CD pipelines (GitHub Actions, pytest) and implementing common OOP design patterns.}
\resumeItemListEnd

\resumeSubheading
{Meta Platforms, Inc.}{Apr. 2019 – Aug. 2019}
{Research Software Engineer Intern}{Menlo Park, CA}
\resumeItemListStart
\resumeItem{Increased system accuracy of an internal content moderation research tool by \qty{25}{\percent} by developing and implementing a collaborative filtering algorithm (Python, scikit-learn, NumPy, SciPy, Polars).}
\resumeItem{Improved recall of image recognition system for content moderation by \qty{5}{\percent} by implementing novel CNN architectures from \qty{5}{\plus} research papers (Python, PyTorch).}
\resumeItem{Streamlined data visualization workflow for internal audits by creating automated tools (Matplotlib, Flask, Python, JavaScript, Tailwind CSS) with direct access to SQL-based database systems and other internal Java-based microservices.
}
\resumeItemListEnd

\resumeSubHeadingListEnd
\vspace{-30pt}

%------------------------
% Projects
%------------------------
\section{Projects}

\resumeSubHeadingListStart

\resumeProjectHeading
{Chemical reaction prediction with ML}{Nov. 2020}
\resumeItemListStart
\resumeItem{Increased reaction prediction accuracy by \qty{30}{\percent} for retro-synthesis tasks by developing a deep-learning-based model (Python, PyTorch) that predicts reaction pathways for chiral amines.}
\resumeItem{Reduced model training time by \qty{50}{\percent} by implementing a distributed training pipeline with PyTorch, Dagster, and Ray on Google Cloud Platform (GCP).}
\resumeItem{Enhanced prediction interpretability by creating visualization tools for reaction pathways (Matplotlib, seaborn, Flask).}
\resumeItemListEnd

\resumeProjectHeading
{Personalized Learning Platform}{Jan. 2018}
\resumeItemListStart
\resumeItem{Enhanced student engagement by \qty{30}{\percent} by implementing a recommendation system using collaborative filtering and content-based filtering techniques (Python, scikit-learn, TensorFlow, pandas).}
\resumeItem{Improved recommendation accuracy by \qty{25}{\percent} by training a hybrid model with TensorFlow, integrating it with a Django back-end and React fron-tend (Django, React, PostgreSQL).}
\resumeItem{Increased platform usability by developing a responsive user interface and automating data collection processes from users and the university's website (JavaScript, HTML, CSS, Selenium).}
\resumeItemListEnd

\resumeSubHeadingListEnd
\vspace{-18.5pt}

%------------------------
% Skills
%------------------------
\section{Skills}

% This section needs some extra padding compared to the others because there's no
% subheading to separate the content (items) from the section title and divider.
\vspace{1pt}

\begin{itemize}[leftmargin=0in, label={}]
    \small{
        \item{
                    \textbf{Languages:} {English, French, Spanish}\\
                    \textbf{Programming:} {Python, C++, Java, JavaScript, SQL, HTML, CSS}\\
                    \textbf{Tools:} {Git, Polars, Docker, Google Cloud Platform (GCP), pandas, MLflow, PyTorch, Dagster, pytest, scikit-learn, NumPy, CI/CD (GitHub Actions), Matplotlib, PostgreSQL, Redis, SciPy, Flask, Tailwind CSS}
              }
    }
\end{itemize}

\end{document}
